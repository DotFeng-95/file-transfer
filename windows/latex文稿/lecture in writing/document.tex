\documentclass{beamer}
 
\usepackage[utf8]{inputenc}
\usepackage{ctex}
 
%Information to be included in the title page:
\title{写作技巧}
\author{冯典}
\institute{湘潭大学\\数学与计算科学学院}
\date{2018年12月21日}
 
 
 
\begin{document}
 
\frame{\titlepage}
 
\begin{frame}
\frametitle{值得注意的几点}
 零零碎碎的注意事项: 
\begin{itemize}
	\item
	格式要求:PDF
	\item
	版面:对文本内容的控制
	\item
	内容:独创、有区分度(换名字、换段落)
	
	
\end{itemize}
\end{frame}

\begin{frame}{表格基本信息的填写}

\begin{itemize}
	\item
	格式要求:美观大方,舒服,对齐(上下,左右)。
	\item
	出生年月:1996年04月(1996.4)\\
	出身日期:1996年04月28日(1996.4.28)\\
	\item
	政治面貌:中共党员,中共预备党员,共青团员(党员,预备党员,团员)
	\item
	部门职务:宣传部部长,宣传部干事(部长,干事)
	\item
	证件照	
\end{itemize}

\end{frame}
 
 
 \begin{frame}{个人简历的填写}
   \begin{itemize}
   	\item
   	格式:首行缩进
   	\item
   	内容:
   	    \begin{itemize}
   	    	\item 
   	    	扣题,不要跑题
   	    	\begin{itemize}
   	    		\item 
   	    		优秀部长,干事(团学工作)
   	    		\item
   	    		优秀党员
   	    		\item
   	    		芙蓉学子自强之星
   	    	\end{itemize}
       	    \item
       	    小标题
   	    	\item
   	    	罗列干货\\
   	    	三段论:xx时候参加了xx活动,做了xx,收获了xx(扣题写)\\
   	    	\quad \qquad (获得了xx奖项,得到了xx评价)
   	    	\item
   	    	忌事情堆砌
   	    	
   	    \end{itemize}
   \end{itemize}
\end{frame}

\begin{frame}{获奖情况的填写}
格式:
  \begin{itemize}
  	\item 
  	标号,写年份
  	\item
  	排序:按奖项的级别,或按获奖时间
  \end{itemize}
\end{frame}

\begin{frame}{小标题的拟写}
小标题是对段落内容的概括与凝练,是最能考察文字基本功的.好的小标题可以画龙点睛(如果水平不够,老老实实分点写).\\

一些例子:
\begin{itemize}
	\item
	不怕吃苦,逆境成长\\
	刻苦学习,创先争优\\
	懂得感恩,服务师生
	\item
	学习刻苦有成绩\\
	乐于奉献有感悟\\
	社会实践有收货\\
	政治素养起表率
	\item
	韶峰初现,湘水源长,三创竞赛把名扬\\
	博学笃行,盛德日新,学术研究需躬行\\
	闻道求知,传承湖湘,社会实践不曾忘
	\item
	创新组织架构,把握工作脉搏\\
	打造优质队伍,助力青年发展\\
	昂扬精神状态,凝心再创新功	
\end{itemize}

\end{frame}

\end{document}