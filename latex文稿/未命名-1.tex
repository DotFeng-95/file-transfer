\documentclass[UTF8]{ctexart}
\usepackage{amsmath}
\numberwithin{equation}{section}
\title{求解不适定问题}
\newtheorem{theorem}{\hspace{2em}定理}[section]
\newtheorem{definition}{\hspace{2em}定义}[section]

\author{DotFeng}
\date{\today}

\bibliography{plain}

\begin{document}

\maketitle


\newpage

\tableofcontents

\newpage
\section{求解不适定问题的正则化方法}
因为偏微分方程的大部分反问题都可以归结为求解第一类算子方程,而第一类算子方程被证明是不适定的.正则化方法就是对不适定方程建立一个稳定的近似解的方法.本节介绍了求解不适定问题的一系列正则化方法,这些方法是求解反问题的基础.
\subsection{预备知识}
\subsubsection{不适定问题}
首先引入不适定问题的定义.

\begin{definition}
设$K:U\subset X\to Y$,其中$X,Y$是赋范空间.方程$Kx=y$叫做适定的,是指$K$是双射,且$K^{-1}:Y\to U$是连续的,否则,就叫做不适定的.
\end{definition}

由以上的定义可知,适定需同时满足下列三个条件:
\begin{itemize}
\item 若对任一个$y\in Y$,至少存在一个$x\in U$,使得$Kx=y$,称为解的存在性(Existence);
\item 设$y_{1},y_{2}\in Y$,若对$x_{1},x_{2}\in U$分别是$Kx=y$对应于$y_{1}\ne y_{2}$的解,则必有$x_{1}\ne x_{2}$,称为解的唯一性(Uniqueness);
\item 解$z$相对于空间偶$(U,Y)$而言是稳定的,称为解的稳定性(Stability),即对于任给的$\epsilon > 0$,存在$\delta(\epsilon)>0$,只要
\begin{equation}
\rho_{U}(x_{1},x_{2})\le \delta(\epsilon),x_{1},x_{2}\in U
\end{equation}
就有
\begin{equation}
\rho_{Y}(y_{1},y_{2})< \epsilon,Tx_{1}=y_{1},Tx_{2}=y_{2}
\end{equation}
\end{itemize}


反之,如果上述三个条件中至少有一个不能满足,就称其为不适定的(ill-posed),由以上定义可知,不适定方程有三种类型:
\begin{itemize}
\item $K$不是满射,即存在$y\in Y$,对任意$x\in U$有$Kx\ne y$;
\item $K$不是单射,即存在$u,v\in U$,且$u\ne v$使得$Ku=Kv$;
\item $K^{-1}$不连续,即方程$Kx=y$的解不连续依赖于数值$f$.
\end{itemize}

在实际问题中,方程右端的数据项一般通过测量获得,精确的数据$y$并不知道,只知道被误差扰动后的测量数据$y^{\delta}$,及其误差水平$\delta >0$,即
\begin{equation}
\Vert y-y^{\delta}\Vert \le \delta,y^{\delta}\in Y. 
\end{equation}
我们的目标是求解扰动后的方程
\begin{equation}
Kx^{\delta}=y^{\delta}.
\end{equation}
通常该方程是不可解的,因为测量数据$y^{\delta}$很可能不在$K$的值域$K(X)$中.于是我们希望找到一个近似解$x^{\delta}\in X$,一方面它能足够接近精确解$x$,另一方面它能连续的依赖于数据$y^{\delta}.$

\subsubsection{正交投影}
本节主要叙述关于正交投影的三个定理.
首先定义一些记号:$X,Y$为Hilbert空间,$L(X,Y)$表示从$X$到$Y$的线性有界算子空间.定义$L(X):=L(X,X)$.
对于$T\in L(X,Y)$,零空间与像空间分别记作$N(T):=\{ \varphi \in X:T\varphi =0\}$和$R(T):=T(X)$
\begin{theorem}
令$U$为$X$的一个凸的线性闭子空间.那么对于每一个$\varphi \in X$,存在唯一的一个向量$\psi \in U$,满足
\begin{equation}
\Vert \psi -\varphi \Vert =\inf_{u\in U}\Vert u-\varphi \Vert.
\end{equation}
$\psi$称为$\varphi$的最佳逼近.$\psi$是$U$中唯一满足此性质的向量,并且
\begin{equation}
\langle \varphi -\psi ,u\rangle =0\ \mbox{对于所有的} u\in U   \label{orthogonal}
\end{equation}
\end{theorem}

\begin{theorem} \label{orthogonal theorem}
令$U\ne \{ 0\}$是$X$的闭线性子空间,令$P:X\to U$表示到$U$的正交投影,其将一个向量$\varphi \in X$映射到其在$U$上的最佳逼近.
则$P$是一个线性算子,且$\Vert P\Vert =1$
满足
\begin{equation}
P^{2}=P \ \mbox{以及} \ P^{\star}=P.
\end{equation}
$I-P$表示到闭子空间$U^{\bot}$上的正交投影.
其中$U^{\bot}:=\{v\in X:\langle v,u\rangle =0,\mbox{对于任意的}u\in U\}$
\end{theorem}

证明:因为$P\varphi =\varphi$对于任意的$\varphi \in U$,得$P^{2}=P$,及$\Vert P\Vert \ge 1.$
在(\ref{orthogonal})中,令$u=P\varphi$,得$\Vert \varphi \Vert ^{2}=\Vert P\varphi \Vert ^{2}+\Vert (I-P)\varphi \Vert ^{2}.$因此$\Vert P\Vert \le 1.$
又因为
\begin{equation}
\langle P\varphi ,\psi \rangle =\langle P\varphi ,P\psi +(I-P)\psi \rangle =\langle P\varphi ,P\psi \rangle
\end{equation}
又$\langle \varphi ,P\psi \rangle =\langle P\varphi ,P\psi \rangle$,得算子$P$是自共轭的.
又由内积的线性性与连续性,易得$U^{\bot}$是$X$的闭线性子空间.
且由(\ref{orthogonal})可得,$(I-P)\varphi \in U^{\bot}$,其中$\varphi \in X.$
更进一步的,$\langle \varphi -(I-P)\varphi,V \rangle =\langle P\varphi ,v \rangle =0$,对所有的$v \in U^{\bot}.$则根据定理(\ref{orthogonal theorem})可知,$(I-P)\varphi$是$\varphi$在$U^{\bot}$中的最佳逼近元.

\begin{theorem} \label{N and R}
如果$T\in L(X,Y)$,那么有
\begin{equation}
N(T)=R(T^{\star})^{\bot} \ \ \mbox{以及} \  \  \overline{R(T)}=N(T^{\star})^{\bot}
\end{equation}
\end{theorem}

证明:如果$\varphi \in N(T)$,则对所有的$\psi \in Y$
有$\langle \varphi ,T^{\star}\psi \rangle =\langle T\varphi ,\psi \rangle =0$
则$\varphi \in R(T^{\star})^{\bot}$.因此,$N(T)\subset R(T^{\star})^{\bot}$
如果$\varphi \in R(T^{\star})^{\bot}$,则对所有的$\psi \in Y$
那么,$0=\langle \varphi ,T^{\star}\psi \rangle =\langle T\varphi ,\psi \rangle$
因此$T\varphi =0$,即$\varphi \in N(T)$,则$R(T^{\star})^{\bot}\subset N(T)$

\subsubsection{Moore-Penrose 广义逆}
考虑算子方程
\begin{equation}
T\varphi =g \label{operator equation}
\end{equation}
其中算子$T\in L(X,Y)$.此时既不假设算子$T$是单射,也不假设$g\in R(T).$在这种假设下,我们先来定义,何为算子方程(\ref{operator equation})的解.这就引出了算子$T$的逆的概念.

\begin{definition}
$\varphi$称为方程(\ref{operator equation})的最小二乘解,如果满足
\begin{equation}
\Vert T\varphi -g\Vert =inf\{\Vert T\psi -g\Vert :\psi \in X\}. \label{least-squares solution}
\end{equation}
$\varphi \in X$被称为方程(\ref{operator equation})的最佳近似解,如果$\varphi$是方程(\ref{operator equation})的最小二乘解,且满足
\begin{equation}
\Vert \varphi \Vert =inf\{ \Vert \psi \Vert :\psi \mbox{是方程(\ref{operator equation})的最佳近似解}\}.
\end{equation}
\end{definition}

\begin{theorem} \label{projection}
用$Q:Y\to \overline{R(T)}$表示投影到$\overline{R(T)}$上面的正交投影.则下列条件是等价的:
%\begin{equation}
\begin{align}
&\varphi \mbox{是}T\varphi =g\mbox{的最小二乘解}.\label{1} \\
&T\varphi =Qg \label{2}\\
&T^{\ast}T\varphi =T^{\ast}g \label{normal equation} 
\end{align}
%\end{equation}
\end{theorem}

证明:因为$\langle T\varphi -Qg,(I-Q)g\rangle =0,$根据定理(\ref{orthogonal theorem})有
\begin{equation}
\Vert T\varphi -g\Vert ^{2}=\Vert T\varphi -Qg\Vert ^{2}+\Vert Qg -g\Vert ^{2}.
\end{equation}
这表明,由(\ref{2})可以推出(\ref{1}).
反之亦然,如果$\varphi_{0}$是一个最小二乘解,则$\varphi _{0}$是函数$\varphi \mapsto \Vert T\varphi -Qg\Vert.$的最小值.
又根据$Q$的定义,$\inf_{\varphi \in X} \Vert T\varphi -Qg\Vert =0,\varphi$必定满足(\ref{2}).

又根据定理(\ref{orthogonal theorem})、(\ref{N and R}),我们有$N(T^{\star})=R(T)^{\bot}=R(I-Q),$
由此可推得等式$T^{\star}(I-Q)$成立.因此由(\ref{2})可得(\ref{normal equation})成立.
反之,如果(\ref{normal equation})成立,则有$T\varphi -g\in N(T^{\star})=R(T)^{\bot}.$
因此$0=Q(T\varphi -g)=T\varphi -Qg.$

方程(\ref{normal equation})称为方程(\ref{operator equation})的正规方程.注意到一个方程的最小二乘解并不一定存在,因为(\ref{least-squares solution})中的下确界并不一定可以取到.由定理\ref{projection}可得,方程(\ref{operator equation})的最小二乘解存在,当且仅当$g\in R(T)+R(T)^{\perp}$

注:如果$\varphi_{0}$是一个最小二乘解,则其并不唯一,若$N(T)$中除去零元素还有其他元素.
此时最小二乘解可写成集合$\{ \varphi_{0}+u:u\in N(T)\}$

\begin{definition}
算子$T$的$Moore-Penrose$广义逆$T^{\dag}$定义在$D(T^{\dag}):=R(T)+R(T)^{\perp}$上,其将$g\in D(T^{\dag})$映到算子方程(\ref{operator equation})的最佳近似解.
\end{definition}

显然,$T^{\dag}=T^{-1}$,如果$R(T)^{\perp}=0$,以及$N(T)=0$.
值得注意的是,并不是对于所有的$g\in Y$,$T^{\dag}g$有定义,假若$R(T)$不是闭集.

令$\widetilde{T}:N(T)^{\perp}\to R(T),\widetilde{T}\varphi :=T\varphi ,$
表示将$T$限制在$N(T)^{\perp}$上.
因为$N(T)^{\perp}=N(T)\cap N(T)^{\perp}=\{ 0\}$
以及$R(\widetilde{T})=R(T),$
则有算子$\widetilde{T}$是可逆的.
则Moore-Penrose 逆可被写成
\begin{equation}
T^{\dag}g=\widetilde{T}^{-1}Qg  
\end{equation}
其中$g\in D(T^{\dag}).$

























\newpage
\section{Tikhonov正则化方法}

\end{document}