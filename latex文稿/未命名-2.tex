\documentclass[UTF8]{ctexart}
\usepackage{amsmath}
\numberwithin{equation}{section}
\title{正交投影的相关理论}
\newtheorem{theorem}{\hspace{2em}定理}[section]
\newtheorem{definition}{\hspace{2em}定义}[section]

\author{DotFeng}
\date{\today}

\bibliography{plain}

\begin{document}

\maketitle


\newpage

\tableofcontents

\newpage
\section{正交投影}
本节主要叙述关于正交投影的三个定理.
首先定义一些记号:$X,Y$为Hilbert空间,$L(X,Y)$表示从$X$到$Y$的线性有界算子空间.定义$L(X):=L(X,X)$.
对于$T\in L(X,Y)$,零空间与像空间分别记作$N(T):=\{ \varphi \in X:T\varphi =0\}$和$R(T):=T(X)$
\begin{theorem}
令$U$为$X$的一个凸的线性闭子空间.那么对于每一个$\varphi \in X$,存在唯一的一个向量$\psi \in U$,满足
\begin{equation}
\Vert \psi -\varphi \Vert =\inf_{u\in U}\Vert u-\varphi \Vert.
\end{equation}
$\psi$称为$\varphi$的最佳逼近.$\psi$是$U$中唯一满足此性质的向量,并且
\begin{equation}
\langle \varphi -\psi ,u\rangle =0\ \mbox{对于所有的} u\in U   \label{orthogonal}
\end{equation}
\end{theorem}

\begin{theorem} \label{orthogonal theorem}
令$U\ne \{ 0\}$是$X$的闭线性子空间,令$P:X\to U$表示到$U$的正交投影,其将一个向量$\varphi \in X$映射到其在$U$上的最佳逼近.
则$P$是一个线性算子,且$\Vert P\Vert =1$
满足
\begin{equation}
P^{2}=P \ \mbox{以及} \ P^{\star}=P.
\end{equation}
$I-P$表示到闭子空间$U^{\bot}$上的正交投影.
其中$U^{\bot}:=\{v\in X:\langle v,u\rangle =0,\mbox{对于任意的}u\in U\}$
\end{theorem}

证明:因为$P\varphi =\varphi$对于任意的$\varphi \in U$,得$P^{2}=P$,及$\Vert P\Vert \ge 1.$
在(\ref{orthogonal})中,令$u=P\varphi$,得$\Vert \varphi \Vert ^{2}=\Vert P\varphi \Vert ^{2}+\Vert (I-P)\varphi \Vert ^{2}.$因此$\Vert P\Vert \le 1.$
又因为
\begin{equation}
\langle P\varphi ,\psi \rangle =\langle P\varphi ,P\psi +(I-P)\psi \rangle =\langle P\varphi ,P\psi \rangle
\end{equation}
又$\langle \varphi ,P\psi \rangle =\langle P\varphi ,P\psi \rangle$,得算子$P$是自共轭的.
又由内积的线性性与连续性,易得$U^{\bot}$是$X$的闭线性子空间.
且由(\ref{orthogonal})可得,$(I-P)\varphi \in U^{\bot}$,其中$\varphi \in X.$
更进一步的,$\langle \varphi -(I-P)\varphi,V \rangle =\langle P\varphi ,v \rangle =0$,对所有的$v \in U^{\bot}.$则根据定理(\ref{orthogonal theorem})可知,$(I-P)\varphi$是$\varphi$在$U^{\bot}$中的最佳逼近元.

\begin{theorem}
如果$T\in L(X,Y)$,那么有
\begin{equation}
N(T)=R(T^{\star})^{\bot} \ \ \mbox{以及} \  \  \overline{R(T)}=N(T^{\star})^{\bot}
\end{equation}
\end{theorem}

证明:如果$\varphi \in N(T)$,则对所有的$\psi \in Y$
有$\langle \varphi ,T^{\star}\psi \rangle =\langle T\varphi ,\psi \rangle =0$
则$\varphi \in R(T^{\star})^{\bot}$.因此,$N(T)\subset R(T^{\star})^{\bot}$
如果$\varphi \in R(T^{\star})^{\bot}$,则对所有的$\psi \in Y$
那么,$0=\langle \varphi ,T^{\star}\psi \rangle =\langle T\varphi ,\psi \rangle$
因此$T\varphi =0$,即$\varphi \in N(T)$,则$R(T^{\star})^{\bot}\subset N(T)$



























\end{document}