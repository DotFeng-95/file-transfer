% !TEX TS-program = pdflatex
% !TEX encoding = UTF-8 Unicode

% This file is a template using the "beamer" package to create slides for a talk or presentation
% - Giving a talk on some subject.
% - The talk is between 15min and 45min long.
% - Style is ornate.

% MODIFIED by Jonathan Kew, 2008-07-06
% The header comments and encoding in this file were modified for inclusion with TeXworks.
% The content is otherwise unchanged from the original distributed with the beamer package.

\documentclass[UTF8]{ctexbeamer}


% Copyright 2004 by Till Tantau <tantau@users.sourceforge.net>.
%
% In principle, this file can be redistributed and/or modified under
% the terms of the GNU Public License, version 2.
%
% However, this file is supposed to be a template to be modified
% for your own needs. For this reason, if you use this file as a
% template and not specifically distribute it as part of a another
% package/program, I grant the extra permission to freely copy and
% modify this file as you see fit and even to delete this copyright
% notice. 


\mode<presentation>
{
  \usetheme{Warsaw}
  % or ...

  \setbeamercovered{transparent}
  % or whatever (possibly just delete it)
}


\usepackage[english]{babel}
% or whatever

\usepackage[utf8]{inputenc}
% or whatever


\usepackage{amsmath}

\usepackage{times}
\usepackage[T1]{fontenc}
% Or whatever. Note that the encoding and the font should match. If T1
% does not look nice, try deleting the line with the fontenc.


\title[\LaTeX 介绍] % (optional, use only with long paper titles)
{\LaTeX 介绍}

\subtitle
{} % (optional)

\author[龙汉清] % (optional, use only with lots of authors)
{龙汉清\\
15数学类韶峰班 }
% - Use the \inst{?} command only if the authors have different
%   affiliation.

\institute[Universities of Somewhere and Elsewhere] % (optional, but mostly needed)
{
 
  数学与计算科学学院\\
  湘潭大学
}

% - Use the \inst command only if there are several affiliations.
% - Keep it simple, no one is interested in your street address.

\date[Short Occasion] % (optional)
{10.20 / 数学院北楼114}

\subject{Talks}
% This is only inserted into the PDF information catalog. Can be left
% out. 



% If you have a file called "university-logo-filename.xxx", where xxx
% is a graphic format that can be processed by latex or pdflatex,
% resp., then you can add a logo as follows:

% \pgfdeclareimage[height=0.5cm]{university-logo}{university-logo-filename}
% \logo{\pgfuseimage{university-logo}}



% Delete this, if you do not want the table of contents to pop up at
% the beginning of each subsection:
\AtBeginSubsection[]
{
  \begin{frame}<beamer>{大纲}
    \tableofcontents[currentsection,currentsubsection]
  \end{frame}
}


% If you wish to uncover everything in a step-wise fashion, uncomment
% the following command: 

%\beamerdefaultoverlayspecification{<+->}


\begin{document}

\begin{frame}
  \titlepage
\end{frame}

\begin{frame}{大纲}
  \tableofcontents
  % You might wish to add the option [pausesections]
\end{frame}


% Since this a solution template for a generic talk, very little can
% be said about how it should be structured. However, the talk length
% of between 15min and 45min and the theme suggest that you stick to
% the following rules:  

% - Exactly two or three sections (other than the summary).
% - At *most* three subsections per section.
% - Talk about 30s to 2min per frame. So there should be between about
%   15 and 30 frames, all told.

\section{\LaTeX 简介}

\subsection[\LaTeX 的主要功能]{\LaTeX 的主要功能}

\begin{frame}{\LaTeX 是什么?}\pause
  % - A title should summarize the slide in an understandable fashion
  %   for anyone how does not follow everything on the slide itself.
\LaTeX 是一种排版系统.\pause


主要功能: \pause
  \begin{itemize}
  \item 生成高质量的科技和数学类文档
    \begin{itemize}
    \item
      学术论文
    \item    
      课堂讲义和课后笔记
    \end{itemize}\pause
  \item
    生成其他简单或复杂种类的文档
    \begin{itemize}
    \item
      信件, 个人简历等
    \item
      书籍, 其他专业类文档
    \end{itemize}
  \end{itemize}
\end{frame}


\begin{frame}{与其他排版软件的比较}

  优势: 
  \begin{itemize}
  \item
    提供专业的版面设计\pause
  \item
    可以方便并且高效率地排版数学公式并且实现自动编号和交叉引用\pause
  \item
    使用的命令简单易懂, 方便修改长文档中复杂的项目\pause
  \item
	鼓励作者按照合理的结构写作\pause
  \item
	免费!
  \item
	$\cdots$
  \end{itemize}
\end{frame}



\begin{frame}{与其他排版软件的比较}

不足: 
  \begin{itemize}
  \item
    很难使用\LaTeX 来写结构不明组织无序的文档\pause
  \item
    debug 麻烦!
  \item
	$\cdots$
  \end{itemize}
\end{frame}


\subsection{下载安装\LaTeX}

\begin{frame}{下载安装}
\includegraphics[height=5cm]{xz1.png}

\end{frame}


\begin{frame}{下载安装}
\includegraphics[height=5cm]{xz2.png}

\end{frame}



\begin{frame}{下载安装}
\includegraphics[height=5cm]{xz3.png}

\end{frame}


\begin{frame}{下载安装}
\includegraphics[height=5cm]{xz4.png}

\end{frame}


\begin{frame}{下载安装}
\includegraphics[height=5cm]{xz5.png}

\end{frame}



\begin{frame}{下载安装}
\includegraphics[height=5cm]{xz6.png}

\end{frame}


\begin{frame}{下载安装}
\includegraphics[height=5cm]{xz7.png}

\end{frame}




\section{\LaTeX 的基础使用}


\subsection{\LaTeX 代码结构}

\begin{frame}{基本结构}

\includegraphics{ys1.png}

\includegraphics{ys2.png}

\end{frame}











\subsection{数学公式}

\begin{frame}{行内公式}

\$ 公式代码 \$


\end{frame}

\begin{frame}{行间公式}

\$\$ 公式代码 \$\$ 

或 $\backslash$[ 公式代码 $\backslash$]

\end{frame}






%
%\section*{Summary}
%
%\begin{frame}{Summary}
%
%  % Keep the summary *very short*.
%  \begin{itemize}
%  \item
%    The \alert{first main message} of your talk in one or two lines.
%  \item
%    The \alert{second main message} of your talk in one or two lines.
%  \item
%    Perhaps a \alert{third message}, but not more than that.
%  \end{itemize}
%  
%  % The following outlook is optional.
%  \vskip0pt plus.5fill
%  \begin{itemize}
%  \item
%    Outlook
%    \begin{itemize}
%    \item
%      Something you haven't solved.
%    \item
%      Something else you haven't solved.
%    \end{itemize}
%  \end{itemize}
%\end{frame}


\end{document}


